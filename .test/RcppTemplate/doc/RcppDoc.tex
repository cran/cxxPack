\documentclass{article}

%\VignetteIndexEntry{R/C++ Interface API Documentation}

\usepackage{vmargin}
\setmargrb{1in}{1in}{1in}{1in}

% Environment for program listings
\newenvironment{program}{\ttfamily\begin{tabbing}
\=xxxx\=xxxx\=xxxx\=xxxx\=xxxx\=xxxx\=xxxx\=xxxx\=xxxx\=xxxx\= \+ \kill \\
}{\end{tabbing}}

% Fix the ``++'' part of C++ (it tends to be too large)
\def\BigC++{C{\raise 0.4ex\hbox{\small ++}}}
\def\C++{C{\raise 0.4ex\hbox{\tiny ++}}}

% Boxing command
\def\boxit#1{\vbox{\hrule\hbox{\vrule\kern3pt
        \vbox{\kern3pt#1\kern3pt}\kern3pt\vrule}\hrule}}

\title{{\bf Rcpp}: A Class Library for R/C++ Object Mapping
  \\ and R Package Development}

\date{October 24, 2009}

\author{Dominick Samperi\footnote{This is an updated version of the
    document {\em Building \R\ Packages that call \C++ functions: \Rcpp:
    R/C++ Interface Classes Version 5.0}, September 20, 2006. 
  I am grateful to
  those who took the time to send kind words of appreciation, and
  I want to thank
Dirk Eddelbuettel, Hin-Tak Leung, Uwe Ligges, Duncan Murdoch, Brian Ripley, and
Paul Roebuck for helpful comments. For
  best results the package {\tt RcppTemplate} should be installed and 
  running in an \R\ session  while reading this guide. 
}}

\newcommand{\R}{{\bf R}}
\newcommand{\Rcpp}{{\bf Rcpp}}
\usepackage{Sweave}
\begin{document}
\maketitle
\begin{abstract}
\noindent A set of \C++ classes that facilitates the process of using
\C++ with the Open Source \R\ statistical software system is described.
The \C++ classes described here model \R\ data structures
like vectors, matrices, factors, dates, data frames, time series,
and functions. Objects of these types can be instantiated from 
pointers (or SEXP's) that
\R\ passes to the \C++ side. The \C++ model objects can also be 
constructed from \C++ data
structures and later returned to \R\ in a format that it recognizes. The
function model can be used to make calls from \C++ to \R\ functions that
take parameters (and return values) of any of the types just mentioned.
\end{abstract}

%\thispagestyle{myheadings}
%\markright{Copyright \copyright\ 2005-2009 Dominick Samperi}
%\tableofcontents

\section{Overview}
\label{sec.overview}

The \R\ system is written in the C language, and it provides a C API for
package developers who have typically coded functions to be called from \R\
in C or FORTRAN. A commonly used interface
(the so-called {\tt .C} interface)
permits only simple objects like vectors and matrices to be passed as
parameters (and returned). A somewhat newer interface called the
{\tt .Call} interface can be used to pass pointers to \R\ objects to
the C/C++ side, but this interface has not yet been fully exploited to
provide an object mapping between \R\ and \C++. 

The purpose of the
\Rcpp\ class library is to provide such a mapping. At present it provides
a mapping between some of the most commonly used data structures on the
\R\ side to \C++ classes. These data structures include vectors, 
matrices, factors, dates, data frames, time series, and functions.
It is possible to call \C++ functions from \R\ with arguments of any of
these types, and \C++ objects can call \R\ functions with arguments of
these types as well. Functions on either side can return values of any
of the supported types.
Currently the \R\ types supported are all S3 classes, but the 
same techniques
could be used to support the newer S4 classes.

When an \R\ function calls a \C++ function using the {\tt .Call} interface
the \C++ side sees each parameter as a pointer to the corresponding
\R\ object. The \C++ class that models each \R\ type has a constructor
that takes this pointer as an argument; it uses it to construct
the model object in the address space of the \C++ module.\footnote{It is
  possible to work with the \R\ object in its own address space but this
  is tricky since \R\ garbage collects.}
The attributes of the \R\ object are effectively copied into the \C++
model where they can be processed naturally using \C++ syntax. The
\C++ classes also have constructors that take \C++ data structures (instead
of a pointer to an \R\ object), and objects constructed using them
can be passed to \R\ just as easily as those constructed from \R\
pointers.

To return values to \R\ or to make calls to \R\ functions the process
is reversed: each \C++ object needs to be transformed into an object
in \R's address space, and a pointer to this object must be passed
to \R. A pointer to an \R\ object as seen on the \C++ side is called
a SEXP,\footnote{The term SEXP comes
  from the Lisp community where it stands for Symbolic EXPression. Unlike
  Lisp everything is not a list (or dotted pair) 
  in \R; vectors are used for faster
  computations. On the other hand, like Lisp a list structure is used
  to represent function expressions.}
 and the protocol just described can be made more explicit using the
 code fragment:
\begin{program}
  RcppExport SEXP CppFunction(SEXP sexp, SEXP fsexp) \{ // C++ declaration.\\
\> RcppFunction f(fsexp,1); // Construct C++ interface f from fsexp (numArgs=1).\\
\> RcppType1 c1(sexp); // C++ constructs object c1 of type RcppType1 from sexp\\
\> $\cdots$ // Do some processing, generate C++ object c2 to be passed to R\\
\> RcppSEXP sp = getSEXP(c2); // Map C++ object c2 (type RcppType2) to SEXP sp.\\
\> f.setNextArg(sp); // Set first and only argument for function call.\\
\> SEXP result = f.callR(); // Call R function.\\
\> RcppType3 c3(result); // Instantiate return value as C++ object.
\end{program}

Here the function {\tt CppFunction} is called from \R\ using the {\tt .Call}
interface with two parameters, {\tt sexp} and {\tt fsexp}. We assume that
the first points to an \R\ object that can be modeled using {\tt RcppType1},
and the second points to an \R\ function that can be modeled using
{\tt RcppFunction}. We assume that the function accepts 1 argument of the
\R\ type mapped to {\tt RcppType2}, and returns a value that can be mapped
to {\tt RcppType3}. The pattern should be clear from the code fragment:
\C++ constructors are used to transform \R\ objects into \C++ objects,
and the overloaded function {\tt getSEXP()} is used to do the reverse
transformation. 

The function {\tt getSEXP()} transforms its input \C++ object into the
corresponding object in \R's address space and returns a pointer (or
SEXP) to this \R\ object. It also returns a
count of the number of addresses in \R's address space that had to
be protected from cleanup by \R's garbage collection. The \Rcpp\ library
takes care of unprotecting these addresses before returning to \R\ so
that the corresponding memory can be freed when no longer in use.

The data structures
currently supported include heterogeneous 
parameter lists (where you would pass in
doubles, reals, strings, etc., with names attached), homogeneous parameter
lists (where all parameter values are numeric and named), 1D vectors, 2D
matrices, dates (both Date and POSIXt types), factors, data frames,
zoo time series objects, and function objects. A summary
of the main classes is given in Table~\ref{tab.mapping}.

\begin{table}[htb]
  \centerline{
    \begin{tabular}{|l|l|l|} \hline
      {\bf Description} & {\bf \R\ class} & {\bf \Rcpp\ class} \\ \hline
      Parameters & {\tt list} & {\tt RcppParams} \\ \hline
      Date & {\tt Date} & {\tt RcppDate} \\ \hline
      Date-Time & {\tt POSIXt POSIXct} & {\tt RcppDateTime} \\ \hline
      Date details & NA & {\tt RcppLocalTime} \\ \hline
      Factor & {\tt factor} & {\tt RcppFactor} \\ \hline
      Data frame & {\tt data.frame} & {\tt RcppFrame} \\ \hline
      Zoo time series & {\tt [zooreg] zoo} & {\tt RcppZoo} \\ \hline
      Frame column & NA & {\tt RcppColumn} \\ \hline
      Function adapter & {\tt function} & {\tt RcppFunction} \\ \hline
      Return list & user-defined & {\tt RcppResultSet} \\ \hline
    \end{tabular}}
  \caption{Mapping between selected \Rcpp\ classes and \R\ classes.}\label{tab.mapping}
\end{table}

Only regular fixed-frequency zoo objects have the {\tt zooreg} class attribute.
Note that a {\tt ts} (fixed-frequency) time series can be transformed 
into an equivalent {\tt zoo} time series using {\tt as.zoo()}, and 
back again using {\tt as.ts()}. Thus fixed-frequency time series can be
handled by encapsulating them in zoo objects. Similarly, a {\tt tseries}
irregularly spaced time series (class {\tt irts}) can be transformed into
a {\tt zoo} time series and back again using {\tt as.zoo()} and
{\tt as.irts()}, so this type of time series object can be handled as well.
The extensible time series class {\tt xts} can be transformed and
managed similarly.

There are two primary use cases. In the first \R\ calls \C++ functions
that do most of the heavy computation at the speed of compiled code,
and \R\ is primarily used to view and perform statistical analysis on
the results. In the second \C++ code makes calls to \R\ functions that
are easily modified in an interactive fashion. In the first case the
goal is faster computations and in the second it is flexibility.

Note that when a call to a \C++ function returns to \R\ all
objects created during the call are destroyed. Thus if persistence is
important it will have to be implemented outside of the \C++ object
system. Interestingly, this can be done to some extent by exploiting
special features of the \R\ system, but we do not discuss this here.

A straight-forward solution is to simply use \R\ to maintain state and to
recreate \C++ objects as needed during the calls. To limit the
amount of copying involved in making repeated calls to \C++ functions
it is possible to work directly with \R's address space,
but this should be avoided since it is inconsistent with the
pass-by-value semantics of \R.

The \Rcpp\ library obviously uses internal information about \R\ and
for this reason there is the danger that changes made to \R\ will
break \Rcpp. Mitigating this risk is the fact that the \R\ types
modeled are well-established and are unlikely to change in ways that
would break existing \R\ scripts. Another mitigating factor
is that the underlying Lisp-like structure of \R\ is simple and unlikely
to change (vectors, lists, attributes, etc.). In the event that a
class like {\tt zoo} time series is changed by the addition of
new attributes, say, it would be relatively easy to update the corresponding
\C++ constructor and {\tt getSEXP()} function to accommodate this
change.\footnote{Given an \R\ object {\tt x}, use {\tt str(x)} to look at
its structure, and use {\tt attributes(x)} to look at its attributes. This
is the information that drives the \C++ constructor and is set by
{\tt getSEXP()}.}

The remainder of this paper is basically a User's Guide for the
\Rcpp\ library. We explain in detail how to build and test \R\ packages
that employ \Rcpp\ under UNIX and Windows. To this end we have
created an \R\ package named {\tt RcppTemplate} that can be used
as a template when creating your own packages. It includes a test
driver ({\tt RcppExample.cpp}) that illustrates how to use all of
the \Rcpp\ classes. 
The process of linking to
external \C++ libraries like {\tt QuantLib} is also explained.

This document and the \emph{Rcpp Quick Reference} can be displayed after
loading the {\tt RcppTemplate} package by issuing the \R\ commands
{\tt showRcppDoc()} and {\tt showRcppQuickRef()}, respectively.

\section{The {\tt RcppTemplate} Package}

The {\tt RcppTemplate} package was designed to be used as a template for
creating packages that make use of \C++ code or external libraries with
the help of the \Rcpp\ interface class library. Basically to create a
new package starting from the template package 
replace the name ``RcppTemplate''
everywhere it occurs with your package name, then modify the 
source, data, demo, and documentation files as needed.\footnote{Remove the
file {\tt inst/doc/RcppDoc.Rnw} and its {\tt Makefile}
before starting work on your own package as this
so-called vignette file triggers extra processing at build time.}

To get started let's download and
run the template package sample code. We will use it to illustrate
how \Rcpp\ is used.
The {\tt RcppTemplate} package is
available at the official \R\ web site,
{\tt http://cran.r-project.org}. By navigating to the packages archive and
going to the {\tt RcppTemplate} package page it an be checked that all
packages have a source file
named {\tt <PackageName>\_<version>.tar.gz}, and most
have a Windows binary with the same name except that ``{\tt .tar.gz}''
is replaced with ``{\tt .zip},'' and a MacOS X version where the suffix
is ``{\tt .tgz}''. To submit your own package to the CRAN archive only
the source archive needs to be sent. This will be unpacked, checked,
compiled, and installed, and all relevant information about the package
will appear on the package page.

The official reference on writing \R\ extensions is
\emph{Writing R Extensions}, available at the CRAN site.
This document should be consulted for additional
information about the C API that is the foundation for \Rcpp. It should be
downloaded along with the source archive for {\tt RcppTemplate} for
later reference.

After installing \R\ the {\tt RcppTemplate} package can be installed
directly from the CRAN archive by using the \R\ command:
\begin{program}
> install.packages("RcppTemplate")
\end{program}
You will be presented with a list of host names from which the package
can be downloaded. Select one that is reasonably close. If you do not
have permission to write to the directory where \R\ was installed you
may be asked for permission to install the package into your personal
file area (say yes, or change the permissions on the \R\ directory).

After installing the package it can be loaded using the {\tt library}
command, and the example function {\tt RcppExample} can be run as follows:
\begin{program}
> library(RcppTemplate)\\
> example(RcppExample)
\end{program}
Actually, the function {\tt RcppExample} takes a long list of parameters
and it would be messy to make an explicit call here. Instead we
run the \R\ code that appears as an example in the manual page for
{\tt RcppExample}. This \R\ code can be viewed by looking at the
man page as follows:
\begin{program}
> ?RcppExample
\end{program}
Scroll down to the ``Examples:'' section where you will
find the actual call; it looks like:
\begin{program}
result <- RcppExample(params, nlist, numvec, ...
\end{program}
{\tt RcppExample} returns a list of objects and assigns this list to
{\tt result}. 

Running the example code defines the variable {\tt result}, and if you
type this variable name on a line by itself a list of names appears.
These are the names of objects on the return list. To view the object
corresponding to a particular name, say {\tt zoo8}, type:
\begin{program}
>  result\$zoo8
\end{program}
The reason why just the object names are printed and not the object contents
(when {\tt result} was entered) will be explained shortly. The object
{\tt result\$zoo8} is of class {\tt zoo}, and if this expression is typed
on a line by itself the contents of this object are displayed.

Let's scroll back up and take a look at the input
parameters on the man page example. The {\tt params} parameter is just
a list of name-value pairs, where the values can be strings,
integers, or dates (if the \C++ code attempts to fetch a value that does
not have the correct type an exception is thrown, and you bounce back
to \R). The {\tt nlist} parameter is very similar, and indeed this
particular type is redundant and should probably be removed.
The nature of the numeric and string vector parameters 
should be obvious.

To assign three consecutive days to {\tt datetimevec} we start with
the current time and add multiples of $60*60*24$, because \R\ measures
time in seconds for this type ({\tt POSIXt}). On the other hand, 
to do the same thing
for the {\tt Date} vector {\tt datevec} this scaling is not
necessary because \R\ measures time in days for this type. On the
\C++ side both date types measure time in the same units, and
mixed comparisons are allowed, with caveats.

The definitions of the factor {\tt myfactor} and the data frame {\tt df}
should be clear (I assume the reader has basic familiarity with \R).

The {\tt zoots} parameter is set equal to an irregularly spaced
time series indexed by Dates.

The {\tt func} parameter is an \R\ function that simply checks the type
of its arguments and returns a vector, and the {\tt hypot} function
computes the distance to the origin. We will show how to call these
functions from \C++ later.

\section{Fetching \R\ Data From the \C++ side}

So what does the \C++ side of this call look like? Something like
this:
\begin{program}
RcppExport Rcpp\_Example(SEXP params, SEXP nlist, SEXP numvec, SEXP numat,\\
\>\> SEXP df, SEXP datevec, SEXP datetimevec, SEXP stringvec,\\
\>\> SEXP fnvec, SEXP fnlist, SEXP zoots, SEXP myfactor) \{\\
\>\> /** Skipping and paraphrasing---see RcppExample.cpp for the truth. */\\
\>\> RcppParams rparms(params);\\
\>\> RcppNumList nl(nlist);\\
\>\> RcppVector<double> vecD(numvec);\\
\>\> RcppMatrix<double> matD(nummat);\\
\>\> RcppFrame frame(df);\\
\>\> RcppFactor factor(myfactor);\\
\>\> RcppDate dateVec(datevec);\\
\>\> RcppZoo  ts(zoots); \\
\>\> /** Omitting lots of stuff. */
\end{program}
All of the input parameters appear as SEXP's, and the corresponding
\C++ objects can be constructed easily from these SEXP's as shown 
here.

We now have a number of \C++ objects that model the
input \R\ data structures. These objects can be used like this:
\begin{program}
\>\> double tolerance = rparam.getDoubleValue("tolerance");\\
\>\> RcppDate startDate = rparam.getDateValue("startDate");\\
\>\> double x = matD(i,j);\\
\>\> string observationStr = factor.getObservationLevelStr(i);\\
\>\> int observationNum = factor.getObservationLevelNum(i);\\
\>\> int numObservations = factor.size();\\
\>\> vector<RcppColumn> cols = frame.getColumns();\\
\>\> RcppDateTime dt = cols[6].getDateTimeValue(row);\\
\>\> double ExcelVal = dt.numExcelPC();\\
\>\> vector<vector<double> > zoodata = ts.getDataMat();\\
\>\> vector<RcppDate> zooindex = ts.getIndDate();\\
\>\> vector<int> perm = ts.getSortPermutation(); \\
\>\> // ordered rows: zooindex[perm[i]], zoodata[perm[i]][j]
\end{program}

The first few examples should be clear. In the case of {\tt RcppFactor}
factor levels are represented by strings or by levels numbers, is in
\R, and the size of a factor is the total number of observations.
Data frames are basically a vector of columns (of type {\tt RcppColumn}),
and data is fetched naturally by column, then by row, as in the case
of the date-time variable {\tt dt}.

Zoo objects can be indexed vectors or matrices (with indexed rows),
and the permutation vector needed to sort the data by the index is
maintained for the case where the user creates the zoo object on the
\C++ side (the data is already sorted when received from the \R\ side
so the permutation is the identity in this case).

The value ExcelVal, if placed into an Excel spreadsheet on a PC 
and formatted as
a date, would display as the date and time corresponding to {\tt dt}. 
This would
not work under MacOS (at least not by default). But this problem is
solved by using {\tt dt.numExcelMac()} instead. This returns the
default value used by Excel on a Mac. There is also
{\tt dt.numJulian()} and {\tt dt.getRValue()}. The first function
returns the julian day number (fractional day part dropped), and the second
function returns the value used by \R\ to represent this date object. All
of these functions work for both {\tt RcppDate} and {\tt RcppDateTime} (the
latter is a subclass of the former). For convenience these functions
are also
implemented at the \R\ script level and included in the package 
namespace---see the man page for {\tt DateNum} when {\tt RcppTemplate} is
loaded.

Incidentally there are a few convenience functions that do things like
return the weekday corresponding to a particular date (values
{\tt RcppDate::Mon}, {\tt RcppDate::Tue}, etc.), that find the next
occurrence of a particular weekday, or that find the n-th occurrence in
a given month.

There is also a useful template function {\tt to\_string(obj)} that
returns the string value of any object that knows how to stream itself.

A Quick Reference on all of the \Rcpp\ classes can be found in
{\tt inst/doc/QuickReference.txt}. It is basically a stripped down version
of {\tt Rcpp.hpp} formatted to show all public methods and the
relationship between the classes. For more detail look at the comments in
{\tt Rcpp.hpp} and {\tt Rcpp.cpp}. Locating these source
files is the next topic.

\section{Package Structure and Source Code}

As mentioned above the source archive (the {\tt .tar.gz} file) can be
downloaded from the CRAN site. It can be extracted into the current
directory using:
\begin{program}
\$ tar -xvzf RcppTemplate\_<version>.tar.gz
\end{program}
Under Windows a {\tt tar} command is available as part of the 
Rtools kit (see Section~\ref{sec.windows}).

The root of the package directory tree will be named {\tt RcppTemplate},
and the main components are {\tt src}: where your \C++ source code
should go, {\tt man}: man page directory, {\tt R}: R scripts that
actually expose your \C++ functions, and {\tt RcppSrc}: contains the
\Rcpp\ source code and should not be modified by you (unless you find
problems). There are also (optional) {\tt demo} and {\tt inst} directories.
The first can contain demo scripts that can be run using the \R\
{\tt demo()} command, and the second directory can contain documentation,
data, and license files that you want to be part of the installed package
(note that the source file directories are NOT part of the installed
package or the Windows {\tt .zip} file).

Other necessary parts of an \R\ package are located in the root
directory (named {\tt RcppTemplate}). These include the
{\tt DESCRIPTION} file and the {\tt NAMESPACE} file, and for
improved portability, a {\tt configure} script.
The {\tt DESCRIPTION}
file gives a brief description of the package, version info, author,
dependencies, etc., and the {\tt NAMESPACE} file lists the functions
that you want exposed by this package (and not confused with
functions of the same name in other packages).

The {\tt NAMESPACE} file shows the name of the dynamic library to
be load when this package is loaded (using the {\tt library} command),
and it also shows the exported \R\ function names. These functions
are usually defined by scripts in the {\tt R} subdirectory that are
run when the package is loaded.

Recall that the function called in the example page was named
{\tt RcppExample}, not {\tt Rcpp\_Example}. The first symbol is
defined in the \R\ script {\tt R/RcppExample.R}, and the second
symbol is defined in the \C++ code
{\tt src/RcppExample.cpp} and made available via the shared library
{\tt RcppTemplate.so} (or {\tt .dll}, as the case may be). The \R\
script is basically a wrapper around the call to the \C++ function,
which looks like this (see {\tt R/RcppExample.R}):
\begin{program}
  result <- .Call("Rcpp\_Example", params, nlist, numvec, ...,\\
               PACKAGE="RcppTemplate") \\
  class(result) <- "RcppExample"
\end{program}

Here {\tt result} is assigned the list of objects returned by the \C++
code in {\tt Rcpp\_Example}, and then this list is given the class
attribute {\tt RcppExample}. The purpose of the class attribute is
to override the default behavior of functions like print() that are
automatically called when the name of a variable is used in certain
contexts. The default behavior of print() is to print everything
recursively, and this would generate too much output for a large
object.

The work-around is to define {\tt print.RcppExample} in the script file
(and to export this name in the  {\tt NAMESPACE} file). The result is that
the command {\tt print(obj)} when {\tt obj} has class {\tt RcppExample}
results in the call {\tt print.RcppExample(obj)}, a specialized function
call for objects of this class. In this case it simply prints the names
of all of the objects on the list returned by {\tt Rcpp\_Example}, and
not the contents of those objects. This is what happens when the
example is run: if after running the example you type {\tt result} on
a line by itself this is equivalent to {\tt print(result)}, which is
dispatched to {\tt print.RcppExample}, and the result is that a list
of object names is displayed, but not the object contents.

As another example, consider a regular (fixed-frequency) zoo object. It
has class {\tt c('zooreg', 'zoo')}, a vector of strings in this case,
where the first string represents the ``derived part'' and the second
is the ``base part.'' The command {\tt print(obj)} for an object of
this type will get dispatched to {\tt print.zooreg(obj)} if this
function is defined, otherwise it will be dispatched to
{\tt print.zoo(obj)} if defined, otherwise it will be dispatched to
{\tt print.default(obj)}. Of course, there is nothing special
about {\tt print} here, and the same class-based dispatch mechanism works for
any functions that are made available via the {\tt NAMESPACE} 
file.\footnote{One improvement made for S4 classes is that
  method resolution works more like it does in the Common Lisp Object
  System (CLOS) where the dispatch mechanism can depend on parameters of
  several different types.}

Finally, let's look at the source code for the \Rcpp\ sample client
application {\tt src/RcppExample.cpp}. The main function there is
named {\tt Rcpp\_Example}, and you will note that there is a 
qualifier {\tt RcppExport} in front of the definition. This ensures
that this function name is available when the package shared library
is loaded. The pattern that should be followed by \C++ modules that are
called by \R\ (and that use \Rcpp) is shown in Figure~\ref{fig.pattern}.

\begin{figure}[htb]
\hspace{.2in}\centerline{\boxit{
\begin{program}
\#include "Rcpp.hpp"\\
using namespace std;\\
using namespace Rcpp;\\
RcppExport SEXP RcppSample(SEXP params, SEXP a) \{\\
\> SEXP rl=R\_NilValue; // Return this when there is nothing to return.\\
\> char* exceptionMesg=NULL;\\
\> try \{\\
\>\> /** Create objects from SEXP's and do some work. */ \\
\>\> RcppResultSet rs; // result set \\
\>\> rs.add("name1", result1); // add items to result set \\
\>\> rs.add("name2", result2);\\
\>\> ...\\
\>\> rl = rs.getResultList(); // done, get return list. \\
\> \} catch(std::exception\& ex) \{\\
\>\> exceptionMesg = copyMessageToR(ex.what());\\
\> \}\\
\> catch(...) \{\\
\>\> exceptionMesg = copyMessageToR("unknown reason");\\
\>\}\\
\> if(exceptionMesg != NULL)\\
\>\> error(exceptionMesg);\\
\> return rl;\\
\}
\end{program}}}
\caption{Use pattern for \Rcpp.}\label{fig.pattern}
\end{figure}

The {\tt using namespace} lines are optional. If you want to reduce
namespace clutter you can remove them and be more explicit:
{\tt std::vector}, {\tt Rcpp::RcppResultSet}, etc. We have already
discussed {\tt RcppExport}, SEXP's, and how to create \C++ objects
from SEXP's. After all of the object creation and calculations are
done the results are returned as a list of objects with the help
of the {\tt RcppResultSet} class (look at {\tt RcppExample.cpp}).

Objects are added to the return list by overloaded {\tt add()} methods
(see the definition of {\tt RcppResultSet} in {\tt Rcpp.hpp}). These
methods call {\tt getSEXP()} to get the corresponding SEXP
pointer, and then add this pointer to the list with the specified
name attached.

The \C++ try/catch mechanism works well in spite of the fact that \R\
is not written in \C++. The \Rcpp\ library does a fair amount of
type checking, bounds checking, and other sanity checking and throws
an exception if there are problems. The message displayed by the catch
clause can then be searched for in the \Rcpp\ source files to help
identify the problem.

The sample file {\tt src/RcppExample.cpp} illustrates how to use most
of the \Rcpp\ classes, and it actually does something non-trivial
with the input data: it computes a schedule of dates between two 
given dates that fall on the n-th occurrence of a specified weekday.

The date conventions illustrated and explained in {\tt RcppExample.cpp}
are as follows
\begin{enumerate}
  \item Dates can be instantiated from \R\ SEXP's as explained above, or
        from the \C++ side by giving month/day/year, and in the 
        case of RcppDateTime, a fraction of day component, so .75
        would translate into 18:00 GMT, for example.
  \item Subtracting two dates yields the number of days between them, and
        this applies to both date types, and even to a mixture of the
        two types (must be interpreted with care due to time zones). The
        result can include a fractional part down to a fraction 
        of a second (in the case of {\tt RcppDateTime}).
  \item Increments to dates are in units of days, so {\tt date++} increases
        {\tt date} by 1 day, and this applies to both date types.
  \item To add n seconds to an {\tt RcppDateTime} object, add
        n/{\tt RcppDate::DAYS2SECS} days.
  \item To inspect the underlying m/d/y or other information about an
        {\tt RcppDate} or {\tt RcppDateTime}, use 
        the {\tt RcppLocalTime} class.
  \item Both date types can be streamed in the usual way.
\end{enumerate}

\section{Calling \R\ Functions from \C++}

There is a demo function in {\tt src/RcppExample.cpp} that illustrates how to
call \R\ functions from \C++. The pattern is as follows. Assume that
you have written an \R\ function named {\tt vol} that accepts two
real parameters, {\tt T} and {\tt K}, and returns a real-valued 
result. To call this function from \C++ we must pass {\tt vol}
as a parameter to the {\tt .Call} function. On the \C++ side we
must know how many parameters the function takes, as well as the
types expected as parameters and return values.\footnote{Like Lisp and
  Smalltalk, \R\ objects are dynamically typed, so types are not
  as easily determined as they are on the \C++ side. It is important that
  \R\ functions that are called from \C++ check their input parameters
  and if the types are invalid use the {\tt stop()} function to 
  issue an error message and terminate.}

If the input SEXP is also named {\tt vol}, then the {\tt RcppFunction}
object is created and used to call the \R\ function like this:\footnote{We
  use {\tt Rprintf} instead of the {\tt iostream} library for debugging
  purposes because {\tt iostream} does not work on some
  systems (Windows), but this comes at a cost: 
  {\tt Rprintf} is not type-safe and if
  you pass it a {\tt std::string} where it expects a {\tt char*} \R\ will
  crash---use {\tt c\_str()}.}
\begin{program}
  RcppFunction f(vol,2);\\
  f.setNextArg(getSEXP(T));\\
  f.setNextArg(getSEXP(K));\\
  SEXP result = f.callR();\\
  RcppVector<double> vec(result);\\
  Rprintf(\verb!"Return value = %lf\n"!, vec(0));
\end{program}
Here the second parameter to the constructor is the number of arguments
expected. The use of an {\tt RcppVector} to capture a single real
value is not natural, but it works. Note that after every {\tt callR} the
arguments are reset so they must be set again before the next {\tt callR}.
Data frames, time series, and other supported types can be passed to
(and returned from) \R\ functions in the same way.

\section{Building and Testing a Package Under UNIX}
\label{sec.unix}

Under UNIX the {\tt RcppTemplate} source package can be built and
installed into a local directory named {\tt Library.test} as follows.
Unpack the source archive, change directory to the parent of
{\tt RcppTemplate}, and run:
\begin{program}
\$ mkdir Library.test\\
\$ R CMD INSTALL -l Library.test RcppTemplate
\end{program}
The directory {\tt Library.test} will contain {\tt RcppTemplate}, the
root of the installed package, and under this will appear a {\tt libs}
subdirectory containing the shared library for 
{\tt RcppTemplate}.\footnote{Because the \Rcpp\ library makes use of \C++ 
templates it sometimes
happens that there are undefined references when you build the pacakge
in this way and there are object and library files from a previous build.
To resolve the problem delete the binary files and build starting
from a ``clean'' state.}

If you leave out the {\tt -l Library.test} part, the package is installed
into the standard place, which is what happens when you use
{\tt install.packages()}. An important advantage of using 
{\tt install.packages()} is that it can determine what
packages this one depends on and download those packages automatically.

To run the package from this test location start \R\ and type:
\begin{program}
> library(RcppTemplate, lib.loc="Library.test")
\end{program}
Use a full path name if you are not immediately above {\tt Library.test}.
Even better, place the library command in a function that is
defined in your {\tt .Rprofile} start-up file:
\begin{program}
  dotemplate <- function() \{\\
    \> library(RcppTemplate, lib.loc="/usr/home/work/Library.test")\\
    \> example(RcppExample)\\
    \}\\
\end{program}
Change the pathname as needed.
This saves a lot of typing and is useful while you are going through
the testing phase.

When you are satisfied that the package functions properly the
package structure can be checked using:
\begin{program}
\$ R CMD check RcppTemplate
\end{program}
This does a thorough consistency test and should be passed before
submitting your package to CRAN. If everything goes well the final
source archive can be created for submission using:
\begin{program}
\$ R CMD build RcppTemplate
\end{program}
This will create {\tt RcppTemplate\_<version>.tar.gz}, where the
version is taken from the {\tt DESCRIPTION} file.

While it is possible to use a traditional {\tt Makefile} to resolve
dependencies and build the package, it is simpler to use a {\tt Makevars}
file. The purpose of this file is to define {\tt PKG\_CPPFLAGS} and
{\tt PKG\_LIBS}, and to build the \Rcpp\ library. The {\tt Makevars} file
(in the {\tt src} subdirectory)
that is automatically generated by the {\tt configure} script in the
{\tt RcppTemplate} package (based on {\tt Makevars.in}) follows:
\begin{program}
PKG\_CPPFLAGS = -I../RcppSrc\\
PKG\_LIBS = -L../RcppSrc -lRcpp\\
RcppLib = ../RcppSrc/libRcpp.a\\
RcppSrc = ../RcppSrc/Rcpp.cpp ../RcppSrc/Rcpp.hpp\\
.PHONY: all\\
all: \$(SHLIB)\\
\$(SHLIB): \$(RcppLib)\\
\$(RcppLib): \$(RcppSrc)\\
\>\> (cd ../RcppSrc; make)
\end{program}
The only part that may require changes based on the needs of your package
are the first two lines, and the
primary purpose of the {\tt configure} script is to make the
necessary changes based on the system configuration where it is run.

If you know what the compiler flags and library locations should be
for your system you can simply modify the first two lines
appropriately and rename the {\tt configure} script to
{\tt configure.save}, say, so that {\tt Makevars} is not overwritten.

Before submitting the package to CRAN the {\tt configure} script will
need to be modified so that any necessary edits are made automatically
and transparently to the package installer.\footnote{This process only
  applies to UNIX and is obscure. Run {\tt autoheader} in the package
  root directory to generate {\tt src/config.hpp.in} from
  {\tt configure.in}. Then run {\tt autoconf} to generate the
  {\tt configure} script and {\tt src/config.hpp}. The package build
  process will start by running the {\tt configure} script, and this
  will create {\tt src/Makevars} by filling in the symbolic values
  that appear in {\tt src/Makevars.in}. The entire process is driven
  by the {\tt configure.in} file, and this is not difficult to follow.
  See the GNU docs on {\tt autoconf} for more information.}
This does not work under
Windows, so instead there is a custom {\tt Makevars.win} file---see
next section.

For example, the directory {\tt inst/doc/QuantLib} contains 
sample {\tt Makevars.in}
and {\tt Makevars.win} files that can be used to link against
the {\tt QuantLib} \C++ quantitative finance library. It also contains
an example distributed with that library. The only changes made to
the example were to replace the {\tt main()} declaration with an
\Rcpp-styple entry point, and to replace all return values with
{\tt R\_NilValue}. See the {\tt README}
file in that directory for more information. The {\tt RQuantLib} package
illustrates how this can be automated and extended.

\section{Building and Testing a Package Under Windows}
\label{sec.windows}

The procedure for building and testing \R\ packages under Windows is very
similar to the one under UNIX, but this requires installing the
following tools:
\begin{itemize}
\item The UNIX tools for \R\ (Rtools)
from http://www.murdoch-sutherland.com/Rtools, 
\item The MinGW GNU compiler,
\item MikTeX (TeX for Windows), 
\item Microsoft's HTML Help Workshop (used to make chm files)
\end{itemize}
The last two items are not essential for development purposes, but
without them package documentation will not be generated in 
certain formats. The first two
items can be installed together because the Rtools kit installation
gives you the option of installing MinGW.

To be sure that all necessary programs are found (and to easily update
the process when you install new versions) you could run a script like
this one ({\tt setrpaths.bat}):
\begin{program}
  set PATH=\verb!c:\Program Files\R\R-2.9.2;%PATH%!\\
  set PATH=\verb!c:\Rtools\bin;%PATH%!\\
  set PATH=\verb!c:\Rtools\MinGW\bin;%PATH%!\\
  set PATH=\verb!c:\Rtools\perl\bin;%PATH%!\\
  set PATH=\verb!c:\Program Files\MikTeX 2.7\miktex\bin;%PATH%!\\
  set PATH=\verb!c:\Program Files\HTML Help Workshop;%PATH%!\\
\end{program}
Of course, the names will have to be modified to match your
system configuration and installed versions. This would typically be run
before each development session.

With the tools installed and the path variable set we can now install
a test version of {\tt RcppTemplate} in {\tt Library.test} just as we
did in the UNIX case (see last section). This should
not be done in directories that have names with spaces in them like 
``{\tt My Documents}'' as this will probably cause the build to fail. It
should also be kept in mind that Windows uses backward slashes while
UNIX uses forward slashes in file name paths (the Rtools binaries accept
slashes in either direction, but Windows tools tend to be less
accommodating).

The package check and package build process is also exactly the same
as in the UNIX case. There is no {\tt configure} script for Windows, and
instead an OS-specific {\tt Makevars.win} file is used. Simply
modify the variables {\tt PKG\_CPPFLAGS} and {\tt PKG\_LIBS} as explained
in the last section based on the needs of your package.

Finally, a Windows binary ({\tt .zip} file) can be created using:
\begin{program}
\$ R CMD \verb!--no-vignettes --force --binary --use-zip RcppTemplate!
\end{program}
Note that it is not necessary to submit a Windows binary version to CRAN.
The Windows binary will be generated by 
CRAN from the submitted source archive
({\tt .tar.gz} file).

Under Windows a package can be installed from a local zip file (instead
of using {\tt install.packages()} to fetch it over the network) by
using the menu option: Packages / Install from local zip files.


\section{Package Demo and Data Files}

Package demo scripts can be inserted into the {\tt demo} subdirectory. These
will typically be larger scripts that do not fit naturally into man pages.
To display the list of demo scripts available and then run one use
commands like:
\begin{program}
  demo(package="RcppTemplate")\\
  demo(SincSurface)
\end{program}
The package has to be loaded for this to work. The demo descriptions
displayed by the first command must be inserted into {\tt demo/00Index}.

The demo script {\tt Peacock.R} illustrates two kinds of dependency that
can occur. It
depends on another package named {\tt ReadImages} that must be installed
(from CRAN), and this package in turn depends on a jpeg shared library 
that must be installed at the OS level.\footnote{Under Windows you may
  need to add the shared library location to your search path (or
  place it in a system directory) so that it will be found at run-time.
  Under UNIX you may need to use the {\tt ldconfig} command to
  accomplish the same objective.}

Package data files, images, license files, and other information that
you want to be part of the installed package should go into the
{\tt inst} subdirectory. The {\tt RcppTemplate} source package contains
{\tt inst/datasets/exam.txt}. When the package is installed the root
directory will contain {\tt datasets/exam.txt}.
To access the file {\tt datasets/exam.txt} in your scripts use code like:
\begin{program}
  myfile <- system.file('datasets', 'exam.txt', package='RcppTemplate')\\
  my.frame <- read.table(myfile, header=TRUE)
\end{program}
For more information see the help page for {\tt read.table()}, for example.

\section{\R\ Package Preparation Checklist}

To conclude, here is a summary of the steps necessary to create an \R\ package:
\begin{itemize}
  \item Place your \C++ (or C or FORTRAN) source code into the {\tt src}
    subdirectory. (Skip this and related steps if your package
    contains only \R\ scripts, no source code to be compiled.)
  \item Set the compiler flags in {\tt Makevars} and {\tt Makevars.win} so
    that any needed external include or library directories will be found
    at build time. For improved portability (on UNIX platforms) use the
    template {\tt Makevars.in} together with a {\tt configure} script.
  \item Update the information in {\tt DESCRIPTION} to show package
    description, version, date, author, dependencies, etc. It is important to
    include dependencies so that {\tt install.packages()} knows what is
    required. A long description can span several lines provided all lines
    after the first begin with a space or a tab.
  \item Insert \R\ scripts that define all of your package functions
    into the {\tt R} subdirectory, and optionally add demo scripts to
    the {\tt demo} subdirectory and update {\tt demo/00Index}.
  \item Update the {\tt NAMESPACE} file to show the package
    shared library name and all exported function names. A start-up message can
    be issued when the library and namespace are initialized by defining
    a function named {\tt .onLoad}.
  \item Insert specially formatted help pages (with {\tt .Rd} suffix) into
    the {\tt man} subdirectory. For example, the {\tt RcppExample} man page is
    {\tt man/RcppExample.Rd}. As was done in this case, it is good practice
    to include an examples section on each man page.
  \item Make sure the package and documentation are consistently defined by
    running the CRAN check function, and when all is well use the build
    command to generate the source archive for distribution.
\end{itemize}

In addition to setting the compiler and linker flags the {\tt configure}
script can take care of other pre-build tasks like copying license
and version information into the {\tt inst} subdirectory. For Windows
an OS-specific {\tt configure.win} can be used (a UNIX-style shell script).
For example, in the {\tt RcppTemplate} package the configure scripts
copy the license and version information for \Rcpp\ from its source
directory to the {\tt inst} subdirectory. After the build this information
can be found in the installed package root directory, and in this way
the end user can determine what version of \Rcpp\ was used to build
the package (important to know if there are problems).

Refer to the ``Writing R Extensions'' manual at the CRAN site for additional
information on package creation, man page formatting, and the \R\ API.

\end{document}

%%% Local Variables: 
%%% mode: latex
%%% TeX-master: "RcppDoc"
%%% End: 
